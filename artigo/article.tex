%  article.tex (Version 3.3, released 19 January 2008)
%  Article to demonstrate format for SPIE Proceedings
%  Special instructions are included in this file after the
%  symbol %>>>>
%  Numerous commands are commented out, but included to show how
%  to effect various options, e.g., to print page numbers, etc.
%  This LaTeX source file is composed for LaTeX2e.

%  The following commands have been added in the SPIE class 
%  file (spie.cls) and will not be understood in other classes:
%  \supit{}, \authorinfo{}, \skiplinehalf, \keywords{}
%  The bibliography style file is called spiebib.bst, 
%  which replaces the standard style unstr.bst.  

\documentclass[]{spie}  %>>> use for US letter paper
%%\documentclass[a4paper]{spie}  %>>> use this instead for A4 paper
%%\documentclass[nocompress]{spie}  %>>> to avoid compression of citations
%% \addtolength{\voffset}{9mm}   %>>> moves text field down
%% \renewcommand{\baselinestretch}{1.65}   %>>> 1.65 for double spacing, 1.25 for 1.5 spacing 
%  The following command loads a graphics package to include images 
%  in the document. It may be necessary to specify a DVI driver option,
%  e.g., [dvips], but that may be inappropriate for some LaTeX 
%  installations.
\usepackage[brazil]{babel}
\usepackage[utf8]{inputenc}
\usepackage{amsfonts}
\usepackage{amssymb}
\usepackage{amsmath}
\usepackage[]{graphicx}

\title{Comparação entre os metodos de registro Demons e Thin Plate Splines aplicados a imagens médicas} 

%>>>> The author is responsible for formatting the 
%  author list and their institutions.  Use  \skiplinehalf 
%  to separate author list from addresses and between each address.
%  The correspondence between each author and his/her address
%  can be indicated with a superscript in italics, 
%  which is easily obtained with \supit{}.

\author{Thiago de Gouveia Nunes\supit{1}
\skiplinehalf
\supit{1}IME-USP, Rua do Matão 1010, São Paulo, Brasil \\
}

%>>>> Further information about the authors, other than their 
%  institution and addresses, should be included as a footnote, 
%  which is facilitated by the \authorinfo{} command.

\authorinfo{E-mail: nunes@ime.usp.br}
%%>>>> when using amstex, you need to use @@ instead of @
 

%%%%%%%%%%%%%%%%%%%%%%%%%%%%%%%%%%%%%%%%%%%%%%%%%%%%%%%%%%%%% 
%>>>> uncomment following for page numbers
% \pagestyle{plain}    
%>>>> uncomment following to start page numbering at 301 
%\setcounter{page}{301} 
 
  \begin{document} 
  \maketitle 

%%%%%%%%%%%%%%%%%%%%%%%%%%%%%%%%%%%%%%%%%%%%%%%%%%%%%%%%%%%%% 
\begin{abstract}
O registro de imagens em tempo real é útil para auxiliar médicos em operações, fundindo em uma única informações
sobre o estado atual do paciente e dados pré operatórios. Esse estudo tem como finalidade a avaliação de algoritmos 
conhecidos na área de registro, o Demons e o ThinPlate Splines, e encontrar pontos em que sejam paralelizáveis, criando
um algoritmo capaz de realizar o registro em tempo real.
\end{abstract}

%>>>> Include a list of keywords after the abstract 

\keywords{registro, demons, tps}

%%%%%%%%%%%%%%%%%%%%%%%%%%%%%%%%%%%%%%%%%%%%%%%%%%%%%%%%%%%%%
\section{INTRODUÇÃO}
\label{sec:intro}  % \label{} allows reference to this section

O registro de imagens já é um problema muito bem firmado e com várias soluções para problemas como a junção de duas 
imagens de um mesmo objeto ou cena que foram adquiridas em ângulos diferentes, ou a reversão de algum tipo de 
distorção, natural ou não, que uma imagem possa vir a sofrer. Os algoritmo de registro são amplamente aplicados
em várias áreas de pesquisa em visão computacional, como em Imagens Médicas, com o objetivo de reverter os
movimentos naturais do corpo entre tomadas de imagens de um paciente, ou em Reconhecimento de Padrões, onde o
registro é aplicado para unificar várias imagens obtidas de um satélite para formar um mapa por exemplo.

Esse trabalho tem como foco a comparação dos métodos \textit{Demons} e \textit{Thin Plate Splines} (TPS) para recuperar
deformações aplicadas a imagens médicas. Ambos são voltados para a recuperação de deformações não-rígidas, algo 
extremamente comum em imagens médicas, o simples ato de respirar entre as tomadas de uma ressonância magnética gera
uma deformação não-rígida na imagem final. 

O algoritmo de registro recebe como entrada duas imagens, a imagem de referência (R) e a alvo (A). A imagem referência
é usada para estimar os parâmetros da transformação que foi aplicada a imagem alvo, e assim a transformação
inversa possa ser calculada e aplicada para retirar a transformação da imagem alvo. Esse processo será explicado
em mais detalhes com relação a cada algoritmo, já que eles variam muito entre si. Normalmente o processo de
descoberta dos parâmetros utiliza pontos conhecidos como características, que mapeiam pontos nas duas imagens que
tenham um grau de informação igual, mas como veremos adiante esse não é o caso do Demons.

Na seção 2, Conceitos, cada método é apresentado. Na seção 3, Resultados, os métodos são comparados tanto em relação 
aos seus resultados quanto as metodologias que aplicam para resolver o problema proposto.

\section{CONCEITOS} 

\subsection{Demons}
	O Demons foi proposto por Jean-Philippe Thirion em 1995. Ele tem como base o modelo de atratores, em que pontos são
definidos das duas imagens e os pontos da imagem alvo são atraídos por pontos da imagem referência usando alguma 
métrica. A métrica mais básica é a de vizinhos próximos, que leva cada ponto ao ponto mais próximo da imagem de 
referência, mas técnicas mais elaboradas como a similaridade de curvatura ou intensidade podem ser utilizadas para 
aumentar a acurácia. Os pontos da imagem alvo então movimentam a imagem até que eles encontrem algum ponto da imagem
de referência.

	O Demons aplica uma dimensão de informação a mais ao modelo de atratores, acrescentando a cada ponto uma direção
associada ao gradiente da imagem. Chamamos cada um desses pontos de Demon. Com essa informação o algoritmo é capaz
de identificar pontos dentro e fora do modelo gerado a partir dos Demons e direcionar a força para empurrá-los ou
atrai-los, respectivamente. Para obtermos o melhor resultado possível adotamos um Demon por pixel/voxel.

\subsubsection{Aproximação matemática da transformação}
	O Demons supõe que a transformação não muda a função de densidade, ou seja, ela só movimenta os pixels e não muda
suas intensidades. A equação seguinte resume isso:
\begin{align}
	i(x(t),y(t),z(t)) = const, \\
\end{align}
	onde $i$ é a intensidade da imagem na posição $x(t),y(t),z(t)$. Derivando (1) temos:
\begin{align}
	\frac{\partial i}{\partial x} \frac{\partial x}{\partial t} +
	\frac{\partial i}{\partial y} \frac{\partial y}{\partial t} = - \frac{\partial i}{\partial t}
\end{align}
	Supondo que as duas imagens que temos diferem de uma unidade de tempo $\partial i/\partial t = 
r-a$, \textit{r} e \textit{a} as intensidades de R e A respectivamente e que a velocidade instantânea $\vec{v} = (dx/dt,dy/dt)$ é aplicada a cada pixel para movê-lo de A para R, 
chegamos a equação:
\begin{align}
	\vec{v}*\vec{\nabla}r = a - r, \ \text{onde} \ \vec{\nabla} r \ \text{é o gradiente de R}
\end{align}
	O inverso da transformação é aproximado por $\vec{v}$. Porém essa equação é instável em relação a norma de $\nabla 
r$. Se a sua norma for muito pequena o Demon em questão é levado para o infinito em alguma direção. Podemos levar em 
conta a diferença dos pixeis de R e A para normalizar a equação (4), obtendo a forma final do Demons:
\begin{equation}
	\vec{v} = \frac{\vec{\nabla}r * (a - r)}{\vec{\nabla}r^2 * (a - r)^2}
\end{equation}

\subsubsection{Implementação}
	Como a formula (5) é degenerada, não podemos calcular o valor de $\vec{v}$ sem utilizar algum artificio. Para tal,
utilizaremos um algoritmo iterativo. Esse algoritmo recebe como entrada as imagens R e A e um campo vetorial
com as dimensões de A que contém uma aproximação da transformação aplicada, esse campo pode ser zero. 
Cada iteração realiza 3 operações:
\begin{itemize}
	\item Para cada Demon em $A_i$, calculamos $\vec{v_i}$, criando um novo campo vetorial $V_i$
	\item Aplicamos um filtro Gaussiano para retirar o ruido introduzido pelo processo em $V_i$
	\item Aplicamos $V_i$ em $A$ para obter $A_{i+1}$;
\end{itemize}
	Esse processo é feito até que $A_i$ convirja à $R$. É importante lembrar que é necessário a
utilização de um algoritmo de interpolação, já que é extremamente provável que o vetor $\vec{v}$
leve os pontos para posições não inteiras. A interpolação trilinear já é suficiente para tal.

\subsubsection{Demons Simetrico}
	O método acima é conhecido como Demons Assimétrico, pois ele só utiliza informações vindas
da imagem referência. No mesmo artigo, Thirion propõe um outro método, conhecido como Simétrico.
Nele a equação para o cálculo de $\vec{v}$ leva o gradiente das imagens $A_i$. Ele obtém resultados
melhores ao custo do cálculo do gradiente de $A_i$ em toda iteração. Sua fórmula é dada por:
\begin{align}
	\vec{v} = \frac{4(a - r)*\vec{\nabla}r|\vec{\nabla}r||\vec{\nabla}a|}
					{(\vec{\nabla}r+\vec{\nabla}a)^2*(\vec{\nabla}r^2 + \vec{\nabla}a^2 + 2(a - r)^2)}
\end{align}

\subsubsection{Implementação Piramidal}
	Podemos criar uma especie de piramide com as imagens para agilizar e melhorar os resultados do Demons.
O Demons é executado utilizando uma pequena parcela de Demons distribuídos de forma uniforme sobre as duas imagens.
O campo vetorial resultante é então passado como entrada para outra instância do Demons que utiliza o dobro de Demons,
até que se chegue em uma instância que utilize 1 Demon para cada pixel das imagens.
	Isso faz com que as primeiras execuções do Demons apliquem transformações iguais em um número maior de pixeis,
assim retirando porções mais globais da deformação. A cada execução nova, o processo se refina mais, retirando porções 
mais locais. No fim somente as pequenas diferenças são retiradas.

\subsection{Thin Plate Splines}
	O Thin Plate Splines (TPS) utiliza um outro paradigma para realizar o registro de imagens. Ele utiliza
características para criar uma função de interpolação que é utilizada para criar a imagem registrada a partir
da imagem referência. Vários algoritmo podem ser utilizados para adquirir as características que serão usadas
pelo TPS, mas não abordaremos esse assunto pois ele foge do escopo do trabalho. Assumimos no inicio da sua
execução que 2 conjuntos de características existem, um para cada imagem, e que uma correspondência entre
eles já é estabelecida.

	Dados as características na imagem referência $(x_i,y_i, i=1,..,n)$ e na imagem alvo $(X_i,Y_i, i=1,..,n)$
o TPS cria uma função que mapeia exatamente cada característica da imagem referência na sua
correspondente na imagem alvo e que é capaz de interpolar os pontos restantes para a imagem final. Para realizar
essa tarefa é utilizada uma função que define uma superfície que sofre a ação de pesos centrados nas
características da imagem referência. A superfície é definida pela seguinte equação:

\begin{align}
	f(x,y) = A_0 + A_1x + A_2y + \sum_{i=0}^n F_i r_i^2 ln r_i^2
\end{align}
Onde $r_i^2 = (x-x_i)^2 + (y-y_i)^2 + d^2$, $d$ é um fator de rigidez da superfície, quanto mais próximo de 
zero $d$ é mais a superfície sofre ação dos pontos de controle, e os pontos $(x_i, y_i)$ são os pontos de controle.

	O TPS deve então determinar os valores das variáveis $A_0, A_1, A_2$ e dos $F_i$. 
Isso é feito através do sistema linear:

\begin{align}
\begin{split}
	\sum_{i=1}^n F_i &= 0 \\
	\sum_{i=1}^n F_ix &= 0 \\
	\sum_{i=1}^n F_iy &= 0 \\
	f(x_1,y_1) &= A_0 + A_1x + A_2y + \sum_{i=0}^n F_i r_{i1}^2 ln r_{i1}^2 \\
	\vdots \\
	f(x_n,y_n) &= A_0 + A_1x + A_2y + \sum_{i=0}^n F_i r_{in}^2 ln r_{in}^2
\end{split}
\end{align}

A equação $\sum_{i=1}^n F_i = 0$ faz com que a soma dos pesos aplicados a superfície seja zero, fazendo com que
ele não se mova. As equações $\sum_{i=1}^n F_ix = 0$ e $\sum_{i=1}^n F_iy = 0$ garantem que a superfície não vai girar.

	Esse sistema deve ser resolvido duas vezes, uma para $f(x,y) = X$ e outra para $g(x,y) = Y$. Com todas as variáveis
encontradas, podemos aplicar as funções de interpolação para desenhar a imagem final.

\section{Experimentos}
	Para os experimentos uma das imagens padrão do software BioImage foi utilizada. Aplicamos três deformações
diferentes nessa imagem. A primeira é dada por:
\begin{align}
\begin{split}
	X &= x + 50(x-x_c)/r \\
	Y &= y + 50(y-y_c)/r 
\end{split} 
\end{align}

Onde $x_c$ e $y_c$ são o número de linhas e colunas divididos por 2 respectivamente e $r = \sqrt{(x-xc)^2 + (y-yc)^2}$.
Essa equação cria uma destorção grande no centro da imagem e que diminui ao chegar perto dos cantos. A próxima é:

\begin{align}
\begin{split}
	X &= x - 8sin(x/32) \\
	Y &= y + 4cos(x/16)
\end{split} 
\end{align}

Ela aplica uma distorção senoidal na imagem referência. A última acrescenta um termo linear na equação acima:

\begin{align}
\begin{split}
	X &= 0.7x - 8sin(x/32) - y + 3 \\
	Y &= 0.9x + 0.8 y + 4cos(x/16) + 5
\end{split} 
\end{align}



% \bibliography{bibliografia}   %>>>> bibliography data in report.bib
% \bibliographystyle{spiebib}   %>>>> makes bibtex use spiebib.bst



\end{document} 

